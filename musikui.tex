%LuaLaTeX
\documentclass[a4paper]{article}
\usepackage{luatexja}
\usepackage{musikui}
\usepackage{listings}
\usepackage[pdfencoding=auto]{hyperref}
\usepackage{longtable}
\usepackage{shortvrb}
\MakeShortVerb{\|}
\newcommand{\Meta}[1]{$\langle$\mbox{}\nobr#1\nobr\mbox{}$\rangle$}
\newcommand{\Note}{\par\noindent ※}
\newcommand{\Means}{:\ }
\newcommand{\nobr}{\nolinebreak[4]}
\newcommand*{\PKN}[1]{\textsf{#1}}

\title{The \PKN{musikui} package v1}
\author{Naoki Kaneko}
\date{2018/04/25}
\begin{document}
\maketitle

This package is for easy expression of arithmetical restorations with \LaTeX.

\begin{center}
\begin{musikui}
\wari{\eaten&\eaten&\eaten&\eaten&\eaten&\eaten&\eaten&\eaten}
{\eaten&\eaten&\eaten}
{\eaten&7&\eaten&\eaten&\eaten}
\musi{\eaten&\eaten&\eaten&\eaten}{4}
\sen
\musi{\eaten&\eaten&\eaten}{3}
\musi{\eaten&\eaten&\eaten}{3}
\sen
\musi{\eaten&\eaten&\eaten&\eaten}{2}
\musi{\eaten&\eaten&\eaten}{2}
\sen
\musi{\eaten&\eaten&\eaten&\eaten}{0}
\musi{\eaten&\eaten&\eaten&\eaten}{0}
\sen
\musi{0}{0}
\end{musikui}
\end{center}

The package is maintained on GitHub:
\begin{itemize}
\item \url{https://github.com/puripuri2100/musikui.sty}
\end{itemize}
\section{Package read}
Read using |\usepackage| command. There is no option.

\begin{quote}
|\usepackage{musikui}|
\end{quote}
\section{Dependent package}
\PKN{graphics} package

\section{License}
The MIT License

\section{Provide command}
\subsection{Commands related to composition}
\begin{quote}
|\kake{|\Meta{multiplicand}|}{|\Meta{multiplier}|}{|\Meta{product}|}|\\
|\wari{|\Meta{dividend}|}{|\Meta{divide}|}{|\Meta{quotient}|}|\\
|\musi{|\Meta{holes}|}{|\Meta{distance from the right end}|}|\\
|\sen|\\
|\bubunsen{|\Meta{length}|}{|\Meta{distance from the right end}|}|
\end{quote}

\subsection{Commands related to holes}
\begin{quote}
|\eaten{|\Meta{numbers etc.}|}|\\
|\noneaten{|\Meta{numbers etc.}|}|\\
|\halfeaten{|\Meta{numbers etc.}|}|\\
|\halfnoneaten{|\Meta{numbers etc.}|}|\\
|\hhalfeaten{|\Meta{numbers etc.}|}|\\
|\hhalfnoneaten{|\Meta{numbers etc.}|}|
\end{quote}

\section{The role of each command}
The role of each command is shown in Table \ref{yakuwari}.
\begin{longtable}[h]{rp{23em}}
\caption{\label{yakuwari}}\\
|\kake|&Outputs \Meta{multiplicand},\Meta{multiplier} and \Meta{product} of multiplication arithmetical restorations calculation.\\
|\wari|&Outputs \Meta{dividend},\Meta{divide} and \Meta{quotient} of division arithmetical restorations calculation.\\
|\musi| &Outputs \Meta{holes},\Meta{distance from the right end}.\\
|\sen|&line\\
|\bubunsen|&Line of the specified length\\
|\eaten|&normal hole\\
|\noneaten| &hole without a line \\
|\halfeaten| &Half the width hole of \verb|\eaten|.\\
|\halfnoneaten| &Hole without a line with half width of  \verb|\eaten|.\\
|\hhalfeaten| &Two holes with |\harleaten| side by side.\\
|\hhalfnoneaten| &|\hhalfeaten| line without a hole
\end{longtable}

\section{Notation}
Use one musikui environment per an arithmetical restorations.
For the representation part of the hole, a hole and a hole (or a number) are connected by  ``|&|".
After using |\kake| or |\wari|, you just line |\musi| and |\sen| like the hole counting you want to express.
An example of division and multiplication is given below.

\begin{longtable}[h]{lr}
\begin{minipage}{0.7\hsize}
\begin{verbatim}
\begin{musikui}
\kake{8&\eaten&6&\eaten}
{\eaten&\eaten}
{\eaten&\eaten&\eaten&\eaten&\eaten}
\musi{\eaten&6&\eaten&\eaten&\eaten}{0}
\musi{\eaten&\eaten&\eaten&6}{1}
\sen
\end{musikui}

\end{verbatim}
\end{minipage}
&
\begin{minipage}{0.4\hsize}
\begin{musikui}
\kake{8&\eaten&6&\eaten}
{\eaten&\eaten}
{\eaten&\eaten&\eaten&\eaten&\eaten}
\musi{\eaten&6&\eaten&\eaten&\eaten}{0}
\musi{\eaten&\eaten&\eaten&6}{1}
\sen
\end{musikui}
\end{minipage}
\\\hline
\begin{minipage}{0.7\hsize}
\begin{verbatim}

\begin{musikui}
\wari{\eaten&\eaten&\eaten&\eaten}
{\eaten&\eaten}
{\eaten&\eaten}
\musi{\eaten&\eaten}{1}
\sen
\musi{1&\eaten}{0}
\musi{\eaten&\eaten}{0}
\sen
\musi{1}{0}
\end{musikui}

\end{verbatim}
\end{minipage}
&
\begin{minipage}{0.4\hsize}
\begin{musikui}
\wari{\eaten&\eaten&\eaten&\eaten}
{\eaten&\eaten}
{\eaten&\eaten}
\musi{\eaten&\eaten}{1}
\sen
\musi{1&\eaten}{0}
\musi{\eaten&\eaten}{0}
\sen
\musi{1}{0}
\end{musikui}
\end{minipage}
\\\hline
\begin{minipage}{0.7\hsize}
\begin{verbatim}

\begin{musikui}
\wari{\eaten{1}&\eaten{0}&\eaten{0}&\eaten{2}}
{\eaten{1}&\eaten{1}}
{\eaten{9}&\eaten{1}}
\musi{\eaten{9}&\eaten{9}}{1}
\bubunsen{4}{0}
\musi{1&\eaten{2}}{0}
\musi{\eaten{1}&\eaten{1}}{0}
\bubunsen{2}{0}
\musi{1}{0}
\end{musikui}

\end{verbatim}
\end{minipage}
&
\begin{minipage}{0.4\hsize}
\begin{musikui}
\wari{\eaten{1}&\eaten{0}&\eaten{0}&\eaten{2}}
{\eaten{1}&\eaten{1}}
{\eaten{9}&\eaten{1}}
\musi{\eaten{9}&\eaten{9}}{1}
\bubunsen{4}{0}
\musi{1&\eaten{2}}{0}
\musi{\eaten{1}&\eaten{1}}{0}
\bubunsen{2}{0}
\musi{1}{0}
\end{musikui}
\end{minipage}
\end{longtable}

\section{Customize}
You can change the value of arithmetical restorations using \verb|\renewcommand*|.

|\renewcommand*{|\Meta{command name}|}{|\Meta{value}|}|

The values whose roles and default values can be changed are shown in Table \ref{kiteiti}.

\begin{longtable}[h]{lll}
\caption{\label{kiteiti}}\\
Command name&Role&Default value \\ \hline
|\musiwidth|&hole width&1.2em\\
|\musiheight|&hole height&0.96em\\
|\musidepth|&hole depth&0.24em\\
|\musihgap|&distance between hole and hole&0.4em\\
|\musivgap|&distance between hole and line&0.4em\\
|\musirule|&line width&0.4pt\\
|\musiopsymbol|&multiplication sign&\verb|$\times$|\\
|\musiwarikakko|&divide symbol&\verb|\Big)|
\end{longtable}

\end{document}
