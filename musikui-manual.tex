\documentclass[dvipdfmx]{article}

\usepackage{musikui}
\usepackage{listings}
\usepackage[dvipdfmx]{hyperref}
\usepackage{longtable}

\title{The musikui package v1.0}
\author{N.K.}
\date{2018/04/25}
\begin{document}
\maketitle

This package is for easy expression arithmetical restorations with LaTeX.

\begin{center}
\begin{musikui}
\wari{\eaten{}&\eaten{}&\eaten{}&\eaten{}&\eaten{}&\eaten{}&\eaten{}&\eaten{}}
{\eaten{}&\eaten{}&\eaten{}}
{\eaten{}&7&\eaten{}&\eaten{}&\eaten{}}
\musi{\eaten{}&\eaten{}&\eaten{}&\eaten{}}{4}
\sen
\musi{\eaten{}&\eaten{}&\eaten{}}{3}
\musi{\eaten{}&\eaten{}&\eaten{}}{3}
\sen
\musi{\eaten{}&\eaten{}&\eaten{}&\eaten{}}{2}
\musi{\eaten{}&\eaten{}&\eaten{}}{2}
\sen
\musi{\eaten{}&\eaten{}&\eaten{}&\eaten{}}{0}
\musi{\eaten{}&\eaten{}&\eaten{}&\eaten{}}{0}
\sen
\musi{0}{0}
\end{musikui}
\end{center}

\section{Package read}
Read using \verb|\usepackage| command. There is no option.

\section{Dependent package}
graphics package

\section{Provide command}
\subsection{Commands related to composition}
\begin{lstlisting}[language=TeX]
\kake{<multiplicand>}{<multiplier>}{<product>}
\wari{<dividend>}{<divide>}{<quotient>}
\musi{<holes>}{<distance from the right end>}
\sen
\bubunsen{<length>}{<distance from the right end>}
\end{lstlisting}

\subsection{Commands related to holes}
\begin{lstlisting}[language=TeX]
\eaten{<numbers etc.>}
\noneaten{<numbers etc.>}
\halfeaten{<numbers etc.>}
\halfnoneaten{<numbers etc.>}
\hhalfeaten{<numbers etc.>}
\hhalfnoneaten{<numbers etc.>}
\end{lstlisting}

\section{The role of each command}
The role of each command is shown in Table \ref{yakuwari}.
\begin{longtable}[h]{rp{20em}}
\caption{\label{yakuwari}}\\
\verb|\kake|&Outputs \verb|<multiplicand>| \verb|<multiplier>| \verb|<product>| of multiplication arithmetical restorations calculation.\\
\verb|\wari|&Outputs \verb|<dividend>| \verb|<divide>| \verb|<quotient>| of division arithmetical restorations calculation.\\
\verb|\musi| &Outputs \verb|<holes> <distance from the right end> |.\\
\verb|\sen|&line\\
\verb|\bubunsen|&Line of the specified length\\
\verb|\eaten|&normal hole\\
\verb|\noneaten| &hole without a line \\
\verb|\halfeaten| &Half the width hole of \verb|\eaten|.\\
\verb|\halfnoneaten| &Hole without a line with half width of  \verb|\eaten|.\\
\verb|\hhalfeaten| &Two holes with \verb|\harleaten| side by side.\\
\verb|\hhalfnoneaten| &\verb|\hhalfeaten| line without a hole
\end{longtable}

\section{Notation}
Use one musikui environment per an arithmetical restorations.
For the representation part of the hole, a hole and a hole (or a number) are connected by  ``\verb|&|".
After using \verb|\kake| or \verb|\wari|, you just line \verb|\musi| and \verb|\sen| like the hole counting you want to express.
An example of division and multiplication is given below.

\begin{longtable}[h]{l|l}
\begin{lstlisting}[language=TeX]
\begin{musikui}
\kake{8&\eaten{}&6&\eaten{}}
{\eaten{}&\eaten{}}
{\eaten{}&\eaten&\eaten{}&\eaten{}&\eaten{}}
\musi{\eaten{}&6&\eaten{}&\eaten{}&\eaten{}}{0}
\musi{\eaten{}&\eaten{}&\eaten{}&6}{1}
\sen
\end{musikui}
\end{lstlisting}
&
\begin{musikui}
\kake{8&\eaten{}&6&\eaten{}}
{\eaten{}&\eaten{}}
{\eaten{}&\eaten&\eaten{}&\eaten{}&\eaten{}}
\musi{\eaten{}&6&\eaten{}&\eaten{}&\eaten{}}{0}
\musi{\eaten{}&\eaten{}&\eaten{}&6}{1}
\sen
\end{musikui}
\\
\begin{lstlisting}[language=TeX]
\begin{musikui}
\wari{\eaten{}&\eaten{}&\eaten{}&\eaten{}}
{\eaten{}&\eaten{}}
{\eaten{}&\eaten}
\musi{\eaten{}&\eaten{}}{1}
\sen
\musi{8&\eaten{}}{0}
\musi{\eaten{}&\eaten{}}{0}
\sen
\musi{\eaten{}}{0}
\end{musikui}
\end{lstlisting}
&
\begin{musikui}
\wari{\eaten{}&\eaten{}&\eaten{}&\eaten{}}
{\eaten{}&\eaten{}}
{\eaten{}&\eaten}
\musi{\eaten{}&\eaten{}}{1}
\sen
\musi{8&\eaten{}}{0}
\musi{\eaten{}&\eaten{}}{0}
\sen
\musi{\eaten{}}{0}
\end{musikui}
\\
\begin{lstlisting}[language=TeX]
\begin{musikui}
\wari{\eaten{}&\eaten{}&\eaten{}&\eaten{}}
{\eaten{}&\eaten{}}
{\eaten{}&\eaten}
\musi{\eaten{}&\eaten{}}{1}
\bubunsen{4}{0}
\musi{8&\eaten{}}{0}
\musi{\eaten{}&\eaten{}}{0}
\bubunsen{2}{0}
\musi{\eaten{}}{0}
\end{musikui}
\end{lstlisting}
&
\begin{musikui}
\wari{\eaten{}&\eaten{}&\eaten{}&\eaten{}}
{\eaten{}&\eaten{}}
{\eaten{}&\eaten}
\musi{\eaten{}&\eaten{}}{1}
\bubunsen{4}{0}
\musi{8&\eaten{}}{0}
\musi{\eaten{}&\eaten{}}{0}
\bubunsen{2}{0}
\musi{\eaten{}}{0}
\end{musikui}
\end{longtable}

\section{Summary}
If all of the above is taken into the drawing, it will be Figure \ref{fig:one} and Figure \ref{fig:two}.

\begin{figure}[htbp]
 \begin{minipage}{0.5\hsize}
  \begin{center}
   \includegraphics[width=60mm]{kake.pdf}
  \end{center}
  \caption{multiplication}
  \label{fig:one}
 \end{minipage}
 \begin{minipage}{0.5\hsize}
  \begin{center}
   \includegraphics[width=60mm]{wari.pdf}
  \end{center}
  \caption{division}
  \label{fig:two}
 \end{minipage}
\end{figure}
\end{document}